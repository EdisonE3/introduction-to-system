\documentclass[10pt,journal,compsoc]{IEEEtran}
\usepackage[utf8]{inputenc}
\usepackage{indentfirst}
\usepackage{graphicx}
\usepackage{listings}
\usepackage{xcolor}
\usepackage{CJK}
\usepackage{amsmath,amssymb,amsfonts}
\usepackage{cite}
\usepackage{amsthm}
\usepackage{geometry}
 \usepackage{listings}
\usepackage{color}
\usepackage{xcolor}


\newcommand\MYhyperrefoptions{bookmarks=true,bookmarksnumbered=true,
pdfpagemode={UseOutlines},plainpages=false,pdfpagelabels=true,
colorlinks=true,linkcolor={black},citecolor={black},urlcolor={black},
pdftitle={The application of asm and taint analysis in Commodity JVMs},%<!CHANGE!
pdfkeywords={Taint Tracking, Dataflow Analysis}}%<^!CHANGE!

\hyphenation{op-tical net-works semi-conduc-tor}

\begin{document}

\title{Student Training Report 2}
% The paper headers

% make the title area
\maketitle
\IEEEdisplaynontitleabstractindextext
\IEEEpeerreviewmaketitle

1) Clone the source code of Linux kernel from the publicly available repo and find the specific kernel version of Ubuntu you installed in Topic 1.
2) Build the kernel image and replace the original image with the one built by you. Make sure your kernel works well.
3) Modify the kernel source code and insert some personal marks or logs into the code, then rebuild a kernel image. Show that your marks or logs work well.

\section{Requirements}

\begin{itemize}
    \item Clone the source code of Linux kernel from the publicly available repo and find the specific kernel version of Ubuntu you installed in Topic 1.
    \item Build the kernel image and replace the original image with the one built by you. Make sure your kernel works well.
    \item Modify the kernel source code and insert some personal marks or logs into the code, then rebuild a kernel image. Show that your marks or logs work well.
\end{itemize}

\section{Steps}

\begin{itemize}
    \item using \textbf{uname -a} to get the version of Ubuntu and the kernel version is 5.15
    \item we clone the source code from \url{https://github.com/torvalds/linux}
    \item We can use busybox to build the Image and launch the Image through qemu
    \item We can modify the file such as setup.c and using function pr_log to add additional logs
\end{itemize}  

\end{document}


