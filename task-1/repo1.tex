\documentclass{article}

\usepackage[colorlinks,linkcolor=red]{hyperref}

\begin{document}
\section{Topic 1}
\subsection{Install Ubuntu (suggested) on your computer or install Ubuntu on a virtual machine.}
Download the Ubuntu image from \url{ubuntu.com} and install it follow the guide.

\subsection{Setup Latex and Git environment on Ubuntu.}
For latex:
\begin{itemize}
    \item sudo agt-get install texlive
    \item sudo apt-get install texlive-full
\end{itemize}

For Git:
\begin{itemize}
    \item sudo agt-get install git
\end{itemize}

\subsection{Create a private repo with Git on Github.}
Sign in the github, and click the button \textbf{New repository}.
Then we can set the porperty of the repo.
After that, we can use commands as following:
\begin{itemize}
    \item git init
    \item git add README.md
    \item git commit -m "init"
    \item git remote add origin \url{github.com/xxx/xxx}
    \item git push
\end{itemize}

\subsection{Submit arbitrary files to the repo with comments using Terminal in Ubuntu.}
\begin{itemize}
    \item git add *
    \item git commit -m "update"
    \item git pull
    \item git push
\end{itemize}

\subsection{Find a way to generate a submission conflict, and solve it.}
We can clone the repo in another place.
Then we can modify the same file.
Then these two file are pushed to the remote.
At this time, one file will be notified that there is a conflict.
We can solve the conflict manually.
Then we can push it successfully.

\end{document}