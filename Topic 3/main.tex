\documentclass[10pt,journal,compsoc]{IEEEtran}
\usepackage[utf8]{inputenc}
\usepackage{indentfirst}
\usepackage{graphicx}
\usepackage{listings}
\usepackage{xcolor}
\usepackage{CJK}
\usepackage{amsmath,amssymb,amsfonts}
\usepackage{cite}
\usepackage{amsthm}
\usepackage{geometry}
 \usepackage{listings}
\usepackage{color}
\usepackage{xcolor}


\newcommand\MYhyperrefoptions{bookmarks=true,bookmarksnumbered=true,
pdfpagemode={UseOutlines},plainpages=false,pdfpagelabels=true,
colorlinks=true,linkcolor={black},citecolor={black},urlcolor={black},
pdftitle={The application of asm and taint analysis in Commodity JVMs},%<!CHANGE!
pdfkeywords={Taint Tracking, Dataflow Analysis}}%<^!CHANGE!

\hyphenation{op-tical net-works semi-conduc-tor}

\begin{document}

\title{Student Training Report 2}
% The paper headers

% make the title area
\maketitle
\IEEEdisplaynontitleabstractindextext
\IEEEpeerreviewmaketitle

We can use the following intruction to launch qemu:
\begin{lstlisting}
    qemu-system-aarch64 -machine virt -cpu cortex-a57 -machine type=virt \
	-m 1024 -smp 4 -kernel arch/arm64/boot/Image --append "rdinit=/linuxrc root=/dev/vda rw console=ttyAMA0 loglevel=8"  -nographic \
	-fsdev local,security_model=passthrough,id=fsdev0,path=./kmodules \
	-device virtio-9p-pci,id=fs0,fsdev=fsdev0,mount_tag=hostshare
\end{lstlisting}

Then we can using following instruction to enable share directory:
\begin{lstlisting}
    mount -t 9p -o trans=virtio,version=9p2000.L hostshare /mnt
\end{lstlisting}

For searching online, we can know that the aarch64 qemu using pl011 as its console device.
To find the address of pl011, we can dump the device tree of qemu:
\begin{lstlisting}
    qemu-system-aarch64 \
    -M  virt,dumpdtb=file.dtb \
\end{lstlisting}

Then we can use dtc to get convert dtb file to dts file.
From the dts file, we can know the address of pl011.

After that, we can use the function ioread32 and iowrite32 to 
read and write the value of the pl011.

For character device, we can use the library <linux/cdev.h> to 
create a driver for such a device. Then we can get the 
minor and major of this device.

Then we can use following command to create a corresponding file: 
\begin{lstlisting}
    mknod "/dev/dev_name" c major minor 
\end{lstlisting}

For a user-level application, it can use system call such as open and read
to interact with the device.

The device driver can modify the structure file_operations to define the behaviour. 

\end{document}


